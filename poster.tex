%%%%%%%%%%%%%%%%%%%%%%%%%%%%%%%%%%%%%
% Multiplicative domain poster
% Created by Nathaniel Johnston
% August 2009
% http://www.nathanieljohnston.com/2009/08/latex-poster-template/
%%%%%%%%%%%%%%%%%%%%%%%%%%%%%%%%%%%%%%

\documentclass[final]{beamer}
\usepackage[scale=1.24,orientation=portrait]{beamerposter}
\usepackage{graphicx}			% allows us to import images

%-----------------------------------------------------------
% Custom commands that I use frequently
%-----------------------------------------------------------

\newcommand{\bb}[1]{\mathbb{#1}}
\newcommand{\cl}[1]{\mathcal{#1}}
\newcommand{\fA}{\mathfrak{A}}
\newcommand{\fB}{\mathfrak{B}}
\newcommand{\Tr}{{\rm Tr}}
\newtheorem{thm}{Theorem}

%-----------------------------------------------------------
% Define the column width and poster size
% To set effective sepwid, onecolwid and twocolwid values, first choose how many columns you want and how much separation you want between columns
% The separation I chose is 0.024 and I want 4 columns
% Then set onecolwid to be (1-(4+1)*0.024)/4 = 0.22
% Set twocolwid to be 2*onecolwid + sepwid = 0.464
%-----------------------------------------------------------

\newlength{\sepwid}
\newlength{\onecolwid}
\newlength{\twocolwid}
\setlength{\paperwidth}{87cm}
\setlength{\paperheight}{120cm}
\setlength{\sepwid}{0.023\paperwidth}
\setlength{\onecolwid}{0.286\paperwidth}
\setlength{\twocolwid}{0.59\paperwidth}
%\setlength{\topmargin}{3cm}
\usetheme{confposter}
\usepackage{exscale}

%-----------------------------------------------------------
% The next part fixes a problem with figure numbering. Thanks Nishan!
% When including a figure in your poster, be sure that the commands are typed in the following order:
% \begin{figure}
% \includegraphics[...]{...}
% \caption{...}
% \end{figure}
% That is, put the \caption after the \includegraphics
%-----------------------------------------------------------

\usecaptiontemplate{
\small
\structure{\insertcaptionname~\insertcaptionnumber:}
\insertcaption}

%-----------------------------------------------------------
% Define colours (see beamerthemeconfposter.sty to change these colour definitions)
%-----------------------------------------------------------

\setbeamercolor{block title}{fg=jred,bg=white}
\setbeamercolor{block body}{fg=black,bg=white}
\setbeamercolor{block alerted title}{fg=white,bg=jred}
\setbeamercolor{block alerted body}{fg=black,bg=dred}

%-----------------------------------------------------------
% Name and authors of poster/paper/research
%-----------------------------------------------------------

\title{On one-step worst-case optimal trisection in univariate bi-objective Lipschitz optimization}
\author{Antanas \v{Z}ilinskas \and Gra\v{z}ina Gimbutien\.{e}}
\institute{Institute of Mathematics and Informatics, University of Vilnius}

%-----------------------------------------------------------
% Start the poster itself
%-----------------------------------------------------------
% The \rmfamily command is used frequently throughout the poster to force a serif font to be used for the body text
% Serif font is better for small text, sans-serif font is better for headers (for readability reasons)
%-----------------------------------------------------------

\begin{document}
\begin{frame}[t]
  \begin{columns}[t]												% the [t] option aligns the column's content at the top
    \begin{column}{\sepwid}\end{column}			% empty spacer column
    \begin{column}{\onecolwid}
      \begin{alertblock}{Abstract}
        \rmfamily{
The bi-objective Lipschitz optimization with univariate objectives is considered. The concept of the tolerance of the lower Lipschitz bound over an interval is generalized to arbitrary subintervals of the search region. The one-step worst-case optimality of trisecting an interval with respect to the resulting tolerance is established. The theoretical investigation supports the previous usage of trisection in other algorithms. The trisection-based algorithm is introduced. Some numerical examples illustrating the performance of the algorithm are provided. \cite{zilinskas.gimbutiene.2015}
	}
      \end{alertblock}
      \vskip2ex

    \begin{block}{The problem and definitions}
	A one-dimensional bi-objective Lipschitz function $f(t)=(f_1(t), f_2(t)), t \in [x_j, x_{j+1}]$, is considered, where
	\begin{equation}
	|f_k(u) - f_k(t)| \le L_k |u - t|, k = 1, 2,  
	\label{eq:2obj_lipsch}
	\end{equation}
	for $\forall u, t \in [x_j, x_{j+1}]$, $L=(L_1, L_2)^T, L_k > 0, k=1, 2$. 

	We will use the following notation:
	\begin{align}
	&f_1(x_i) = y_i, f_2(x_i) = z_i, i=j, j+1,\\
	&\delta y = |y_j - y_{j+1}|, \delta z = |z_j - z_{j+1}|, \\
	&C = \frac{1}{2}\sqrt{L_1^2+L_2^2}, \\
	&\Psi = \{f(\cdot): \text{function } f(\cdot) \text{ satisfies (\ref{eq:2obj_lipsch}) and }\\\nonumber&f(x_i)=(y_i, z_i), i=j, j+1\}.
	\end{align}

	\textbf{Definition 1.}
	For any interval $[x_j, x_{j+1}]$ with known $f(x_j), f(x_{j+1})$
	\begin{align}
	&\Delta((x_j, x_{j+1}), (f(x_j), f(x_{j+1}))) = \\
	\nonumber
	&C\max\left((x_{j+1}-x_j) - \frac{\delta y}{L_1}, (x_{j+1}-x_j) - \frac{\delta z}{L_2}\right).
	\end{align}

	\textbf{Definition 2.}
	For an interval $[r_1, r_2], r_1\le r_2,$ such that $[r_1, r_2] \subseteq [x_j, x_{j+1}]$, and known $f(x_j), f(x_{j+1})$, the tolerance is defined as:
	\begin{align}
	&\bar{\Delta}((r_1, r_2, x_j, x_{j+1}), (f(x_j), f(x_{j+1})))=\\\nonumber
	&\max_{f(\cdot)\in\Psi}\Delta((r_1, r_2), (f(r_1), f(r_2))).
	\end{align}
    \end{block}
      \vskip2ex
    \end{column}

    \begin{column}{\sepwid}\end{column}			% empty spacer column
    \begin{column}{\twocolwid}				% create a two-column-wide column and then we will split it up later
      \begin{columns}[t,totalwidth=\twocolwid]	% split up that two-column-wide column
        \begin{column}{\onecolwid}\vspace{-.69in}
    \begin{block}{Worst-case optimal trisection}
	We are interested in trisecting the interval $[r_1, r_2]$, $x_j\le r_1 < r_2 \le x_{j+1}$, using points \mbox{$\tilde{a}, \tilde{b} \in (r_1, r_2), \tilde{a} < \tilde{b}$}. The worst-case optimality criterion for the choice of the points is defined as follows:
	\begin{align}
	\label{eq:trisection_problem}
	&(\tilde{a}, \tilde{b}) =\\
	&\arg \min_{a, b} \bar{\Delta}^*((a, b, r_1, r_2, x_j, x_{j+1}), (f(x_j), f(x_{j+1}))),\\\nonumber
	&\bar{\Delta}^*(\cdot) = \max_{f(\cdot)\in\Psi} \max[\bar{\Delta}((r_1, a, x_j, b), (f(x_j), f(b))),\\\nonumber
	&\bar{\Delta}((a, b, x_j, x_{j+1}), (f(x_j), f(x_{j+1}))),\\\nonumber
	&\bar{\Delta}((b, r_2, a, x_{j+1}), (f(a), f(x_{j+1})))
	].
	\end{align}
    \end{block}
        \end{column}
        \begin{column}{\onecolwid}\vspace{-.69in}

        \begin{block}{Experimental results}
        \end{block}
        \end{column}
      \end{columns}
      \vskip2.5ex

      \begin{alertblock}{Pseudocode}		% an ACTUAL two-column-wide column
      \end{alertblock}

      \begin{columns}[t,totalwidth=\twocolwid]
        \begin{column}{\onecolwid}
        \begin{block}{Theorems}
	\textbf{Lemma 1.}
	Denote 
	\begin{align}
	\label{eq:beta}
	&\beta = \beta((x_j, x_{j+1}), (f(x_j), f(x_{j+1}))) =\\\nonumber
	&(x_{j+1}-x_j) - \min\left(\frac{\delta y}{L_1}, \frac{\delta z}{L_2}\right).
	\end{align}
	Then 
	\begin{align}
	&\bar{\Delta}((r_1, r_2, x_j, x_{j+1}), (f(x_j), f(x_{j+1})))= \\\nonumber
	&=C \times
	\begin{cases}
	\label{eq:delta_general}
	r_2 - r_1, & \text{if } r_2 - r_1 \le \beta, \\
	\beta & \text{otherwise.}
	\end{cases}
	\end{align}

	\textbf{Theorem 1.}
	Let $x_j \le r_1 < r_2 \le x_{j+1}$ and $\beta$ defined by (\ref{eq:beta}). The worst-case optimal division points $\tilde{a}$ and $\tilde{b}$ of $(r_1, r_2)$, solving (\ref{eq:trisection_problem}),  are defined as follows:
	\begin{enumerate}[1)]
	\item $\tilde{a}=r_1+\frac{1}{3}(r_2-r_1), \tilde{b}=r_1+\frac{2}{3}(r_2-r_1)$, if $\frac{1}{3}(r_2-r_1) < \beta$;
	\item an arbitrary choice $\tilde{a} \in (r_1, r_2), \tilde{b} \in (r_1, r_2)$, otherwise.
	\end{enumerate}
	Then $\bar{\Delta}^*(\cdot)$ equals $\frac{C}{3}(r_2-r_1)$ in the first case, and $C\beta$ in the second.
        \end{block}

        \end{column}
        \begin{column}{\onecolwid}

     	\begin{block}{Conclusions}
     	\end{block}
    	\begin{block}{References}
    	    \bibliographystyle{siam}
    	    \bibliography{poster}   % name your BibTeX data base
    	\end{block}

      \end{column}
    \end{columns}
  \end{column}

  \begin{column}{\sepwid}\end{column}			% empty spacer column
 \end{columns}
\end{frame}
\end{document}
